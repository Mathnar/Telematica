\documentclass{report}% use option titlepage to get the title on a page of its own.

\usepackage{amsthm}
\theoremstyle{plain}

\begin{document}



\chapter*{2\\IL LIVELLO APPLICATIVO} 

\section*{Introduzione}

In questo capitolo verranno analizzati alcuni protocolli facenti parte di entrambi i modelli standard di rete informatica: il modello OSI ed il modello TCP/IP. \\
Benché lo standard internazionale per la comunicazione sia rappresentato dal modello OSI, nella pratica comune il più utilizzato è il modello TCP/IP, caratterizzato da un numero inferiore di livelli di rete e da una conseguente maggiore praticità. \\
I livelli OSI riguardati da questa modifica sono i livelli che andremo ad analizzare in questo capitolo, definendoli e visionandone i principali servizi, a partire dal più esterno livello applicativo.\\
Il livello applicativo (o applicazione) è il settimo ed ultimo livello nel modello OSI, ed il quarto ed ultimo livello nel TCP/IP. Il parallelismo di tale livello nei due standard sarà motivo di studio nelle sezioni successive.

\section*{2.1 Funzionalità}

Il livello applicazione è un livello astratto che ha la funzione di specificare all'interno di una rete le interfacce ed i protocolli di comunicazione utilizzati e più in generale di fornire servizi alle applicazione degli utenti della rete.\\\\
Quando due programmi devono comunicare tra loro è necessario prima di tutto che esista una connessione logica tra le due entità logiche (mittente e ricevente). Per mezzo di essa si può andare a gestire i quattro livelli dello stack TCP/IP tramite un insieme di istruzioni (o funzioni) denominate API (Application Program Interface) per mezzo delle quali è possibile aprire e chiudere connessioni ed inviare e ricevere dati. \\Durante una trasmissione, ogni strato dello stack aggiunge un header al pacchetto dati che identifica il messaggio da trasmettere. Questa operazione è chiamata incapsulamento. Per mezzo di essa è possibile identificare il dato effettivo e l'header aggiunto dallo strato attuale come un unico pacchetto dati che verrà successivamente passato allo strato sottostante. Il pacchetto dati risultante dal passaggio nello stack verrà poi trasmesso e ricevuto. Il computer o l'applicazione ricevente dovrà poi effettuare  l'operazione inversa, estraendo ad ogni livello l'header necessario a dirigere l'operazione di ricezione.
\\\\Il primo e più esterno strato dello stack è il livello applicativo, il quale ha lo scopo di standardizzare la comunicazione. Nel TCP/IP infatti, il livello applicativo contiene i protocolli e le interfacce di comunicazione usati nelle trasmissioni processo a processo per completare le varie richieste dei programmi, svolte attraverso il protocollo internet (IP), all'interno di una rete. Tra i protocolli più comunemente usati troviamo HyperText Transfer Protocol (HTTP), File Transfer Protocol (FTP), Domain Name System (DNS) e Simple Mail Transfer Protocol (SMTP).\\
Un esempio di concreto utilizzo dei protocolli si ha ogni volta che richiediamo al nostro browser di accedere ad internet o di caricare o scaricare qualcosa da esso. Il livello applicativo per completare la richiesta, chiamerà nel primo caso il protocollo HTTP mentre nel secondo il protocollo FTP.\\
Per far arrivare all'utente i dati richiesti secondo lo standard TCP/IP, il livello applicazione deve passare i dati al livello trasporto attraverso degli indirizzi ( o interfacce ) logici chiamati porte. Sarà poi il livello di trasporto ad occuparsi della ricezione finale da parte dell'utente.\\
L'utilizzo delle porte rende molto più semplice al livello di trasporto, capire che tipo di dato gli sta venendo passato. Se ad esempio venisse inviato un dato sulla porta 25, significherebbe che il dato riguarda una richiesta e-mai. Se invece fosse stato mandato sulla porta 21 o 22, sarebbe stato un dato FTP.\\\\ 
Il livello applicazione nel TCP/IP non descrive nella comunicazione, specifiche regole o formati di dati che le applicazioni devono invece tenere di conto durante le trasmissioni. Per questo motivo nella specifica iniziale dello standard è fortemente raccomandato di seguire il robustness principle (principio di robustezza). \\

\newtheorem*{theorem*}{Principio di robustezza}
\begin{theorem*}
(o legge di Postel): “be conservative in what you do, be liberal in what you accept from others”.
 \end{theorem*} 
 In altre parole, quando viene inviato un messaggio 			ad un altro computer o applicazione, questo deve essere perfettamente conforme alla specifica. In caso invece di messaggi in entrata non conformi, questi devono essere accettati fintanto che il loro significato è chiaro.\\\\
Nel modello OSI il livello applicativo è costituito, come nel TCP/IP, da un insieme di protocolli ed interfacce di comunicazione, e come  nel 	TCP/IP troviamo tra i protocolli più comunemente usati HTTP, FTP, DNS e SMTP. Una sostanziale differenza però la si trova, come anticipato, dalla struttura stessa del modello. Il livello applicativo del TCP/IP comprende sia il livello applicativo del modello OSI, sia i livelli presentazione e sessione. Questa maggiore modularità degli strati dello standard OSI è dovuta ad una più elevata distribuzione delle funzioni tra essi. Il livello applicazione del modello OSI infatti, si occupa solo di verificare la effettiva disponibilità di un partner per la comunicazione e delle risorse necessarie ad essa, lavorando direttamente col software applicativo. Una volta appurata la presenza di un partner e la disponibilità di risorse, il pacchetto dati composto dal dato da scambiare e dall'header aggiunto dal livello applicazione viene passato al sottostante livello presentazione.
\section*{2.2 Livelli Presentazione e Sessione}
Il livello presentazione è il sesto livello del modello ISO/OSI ed ha la funzione di assicurarsi che il dato ricevuto sia compatibile con le risorse disponibili alla comunicazione. Questo avviene tramite una traduzione del dato in una forma accettabile (quindi comprensibile) dal livello applicazione e dai livelli sottostanti.\\Il sesto strato offre inoltre importanti servizi quali la formattazione, la codifica, la criptazione e la compressione di dati. Se ad esempio si volesse convertire un file di testo scritto in Extended Binary Coded Decimal Interchange Code (EBCDIC) in un file scritto in  American Standard Code for Information Interchange (ASCII), tale conversione avverrebbe a questo livello. Un altro esempio si ha quando si invia un file di testo dal contenuto sensibile (come una password), ed è dunque necessario che il messaggio venga criptato. Tale operazione avviene proprio al livello presentazione. Una volta finita l'elaborazione del pacchetto dati, questo viene passato allo strato inferiore, il livello sessione.\\\\
Il livello sessione è il quinto livello del modello OSI e fornisce i meccanismi per stabilire, gestire ed infine concludere la connessione tra una applicazione locale ed una remota, svolgendo inoltre un'operazione di verifica, per garantire la corretta consegna del pacchetto dati. È solo a questo punto dello stack infatti, dopo che il livello applicativo e presentazione hanno adattato il pacchetto dati alla comunicazione, che avviene l'effettiva connessione. A questo livello vengono gestite anche le funzioni di autenticazione e autorizzazione e viene fornito il servizio di session restoration (checkpointing and recovery). Quest'ultimo servizio proverà, in caso di perdita di connessione, a ripristinarla riavviando se necessario la comunicazione.\\ Una testimonianza concreta del passaggio dal livello sessione si ha durante una video chiamata, durante la quale è essenziale che il video e l'audio siano sincronizzati per evitare problemi di lip synch (problemi di sincronizzazione del movimento delle labbra con l'audio). Se il quinto strato è ben progettato, la persona sullo schermo corrisponderà sempre all'audio in uscita, ed in caso di perdita di connessione, il meccanismo di session restoration, proverà a ristabilirla. \\Date le sue caratteristiche,  il livello sessione viene tipicamente implementato negli ambienti applicativi che fanno uso delle chiamate a procedura remote (RPCs), ma soprattutto nei Web browsers. In questi ultimi utilizza protocolli come lo Zone Information Protocol (ZIP) ed il Session Control Protocol (SCP).
\section*{2.3 Evoluzioni delle applicazioni Internet e Web}
Dalla nascita di ARPANET negli anni '60 si impose nel mondo un nuovo modo di comunicare: la rete internet. Sebbene inizialmente i fini fossero puramente militari, con la fine della guerra fredda l'esercito si disinteressò dello strumento, lasciandolo nel pieno controllo delle università dove divenne un utile strumento per scambiare conoscenze scientifiche e per comunicare. Negli stessi anni Tim Berners-Lee creò un'architettura che semplificò drasticamente l'utilizzo della rete, ormai rinominata Internet: il World Wide Web.\\Nel 1991, sempre lo stesso Tim Berners-Lee, creò il primo sito web, avviando una rivoluzione che oggi chiamiamo Web 1.0.\\\\
Il Web 1.0 era l'internet dei contenuti, caratterizzato da siti web semplici e statici dai quali era possibile solo accedere a contenuti senza poterli modificare o aggiungere. Le pagine web erano scritte da una ristretta cerchia di persone ed erano gremite di collegamenti ipertestuali (link) ad altre pagine, in modo da dare all'utente più libertà di movimento possibile all'interno della rete. È col Web 1.0 che nasce il concetto di applicazione internet (web application), cioè un qualsiasi programma di tipo client-server che viene eseguito dal client tramite un web browser. Nel Web 1.0, il client, rappresentato dal browser, mezzo di uno o più web server accede a pagine web statiche (figura web client ->network->web server->web pages). I browser utilizzati per la navigazione erano molto semplici poiché l'unica cosa che dovevano tradurre era l'HTML. Nel Web 1.0 non era inoltre prevista alcun tipo di separazione tra i dati e la loro rappresentazione, era tutto semplicemente ammassato nelle pagine web, rendendo queste ultime molto fragili. Se ad esempio doveva essere cambiata anche solo una parola, si doveva andare a modificare l'intera pagina.  \\Le comunicazioni erano gestite, come adesso, dal protocollo HTTP il quale ha la caratteristica di essere un protocollo "senza stato" (state-less protocol). Questo significa che ogni richiesta da client a server, è trattata in modo indipendente dalle altre, e di conseguenza ogni informazione scambiata tra browser e server alla fine della trasmissione, viene persa. Se ad esempio avessimo messo qualcosa nel carrello di un qualche sito di acquisti, ed avessimo poi navigato su altri siti o chiuso e riaperto il collegamento, una volta tornati al nostro sito di acquisti, avremmo trovato il carrello vuoto. Questo problema venne risolto sempre in quel periodo con l'invenzione dei cookies. Per mezzo di essi, ogni volta che l'utente effettuava una richiesta, il browser, se necessario, inviava al server i cookies per notificarlo di connessioni precedenti.\\\\
Col progredire degli anni e del numero degli utenti, si rendeva sempre più necessaria la possibilità di poter interagire con i contenuti, tanto da spingere i ricercatori a cercare un modo per far passare la rete da statica a dinamica. Il mutamento iniziò grazie all'ausilio dei nuovi linguaggi di programmazione (come PHP), per mezzo dei quali iniziarono a crearsi i primi blog sui quali era possibile inserire dei commenti. Questo primo cambiamento comportò l'avvento delle community e venne identificato come Web 1.5. \\Di lì a pochi anni la rete si espanse in modo esponenziale con l'introduzione dei Wiki e dei Social Network, facendo passare l'interattività con l'utente in primo piano e facendo nascere così il Web 2.0.











\end{document}